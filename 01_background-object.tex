\chapter{プロジェクトの背景と目的}

\section{背景}
学習指導要領が平成31年?に改定され,
学生は主体的な姿勢で学ぶ必要がある,などと強調された.

主体的な姿勢を身に着けることを目的とした学びをアクティブラーニングと呼び,
これは主に生徒複数人が集まって,ある課題の達成を目指すグループワークによって実現される.

その手法の一つにグループディスカッションがあり,これは生徒に対するフィードバックが不十分か不透明であったり,主体的に参加しない生徒が存在するなどの課題があった[課題を紹介していた文書].

先行研究[?]では,何かしていた.
ナンタラら[2]は,教員の評価をなるべく公平に保つべく,評価表を作成し生徒と共有していた?


\section{目的}
中高生のグループワーク促進が目的.
生成AIを用いて話し合いの様子を評価し,
それを生徒に渡して自省を促したり,
グループワークやディスカッションを考えるきっかけにして,より楽しく学びを得られるようにすることを目指す.
