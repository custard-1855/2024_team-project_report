\chapter{概要}
平成31年に行われた学習指導要領の改定で,今後社会で生き抜くためには学習に対する生徒の主体性が必要とされ,その獲得にはアクティブラーニングが有効とされている.
アクティブラーニングの一手法であるディスカッションは,学生の議論に対する態度や貢献などの評価や,議論の進行のばらつきに課題があった.
私たちは評価の公平性と透明性の向上と,生徒の積極的な議論への参加のため,GPT4モデルを利用した議論の評価とそのフィードバックを行う.
それによりディスカッションが活性化し,学習効果の増大,ひいては主体的な学習姿勢の獲得につながると期待する.
