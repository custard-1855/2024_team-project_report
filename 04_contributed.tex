開発 最初の発表まで
- 個人サーベイ テーマ決定
 - 全員教育関連でサーベイしてた
 - 経緯は議事録を参照
- チームサーベイ アプリ内容の議論
 - 教育の問題と既存の解決方法,生成AI活用事例を調査
 - 候補は多数出たが議事録の作成と評価に決定
 - 経緯は議事録へ
- 役割分担: 僕は機能の開発のはず
- スケジュール作成: そこまで無理のないスケジュールだったが,思ったよりAPIの選定やAmivoiceのAPIの利用に手間取り,機能面の開発が遅れた


夏季休暇開始から中間まで
- 毎週水曜前後にミーティング
 - 殿村君の提案 これはかなり良かった
 - ここを目標に小さな成果を積み上げられた
- 僕: 録音,認識などの機能を実装
 - 録音
  - 議事録を音声認識で生成するべく,音声の入手から実装
  - 
 - 
ajaxのエラー解決がその後も大変だった


中間発表後から年末まで
- 評価機能を実装
- 評価表示機能を実装
- エラーの解決

